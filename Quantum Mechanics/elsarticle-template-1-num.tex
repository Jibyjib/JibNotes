%% This is file `elsarticle-template-1-num.tex',
%%
%% Copyright 2009 Elsevier Ltd
%%
%% This file is part of the 'Elsarticle Bundle'.
%% ---------------------------------------------
%%
%% It may be distributed under the conditions of the LaTeX Project Public
%% License, either version 1.2 of this license or (at your option) any
%% later version.  The latest version of this license is in
%%    http://www.latex-project.org/lppl.txt
%% and version 1.2 or later is part of all distributions of LaTeX
%% version 1999/12/01 or later.
%%
%% The list of all files belonging to the 'Elsarticle Bundle' is
%% given in the file `manifest.txt'.
%%
%% Template article for Elsevier's document class `elsarticle'
%% with numbered style bibliographic references
%%
%% $Id: elsarticle-template-1-num.tex 149 2009-10-08 05:01:15Z rishi $
%% $URL: http://lenova.river-valley.com/svn/elsbst/trunk/elsarticle-template-1-num.tex $
%%
%%\documentclass[preprint, 12pt]{elsarticle}

%% Use the option review to obtain double line spacing
 \documentclass[preprint, review,12pt]{elsarticle}

%% Use the options 1p,twocolumn; 3p; 3p,twocolumn; 5p; or 5p,twocolumn
%% for a journal layout:
%% \documentclass[final,1p,times]{elsarticle}
%% \documentclass[final,1p,times,twocolumn]{elsarticle}
%% \documentclass[final,3p,times]{elsarticle}
%% \documentclass[final,3p,times,twocolumn]{elsarticle}
%% \documentclass[final,5p,times]{elsarticle}
%% \documentclass[final,5p,times,twocolumn]{elsarticle}

%% if you use PostScript figures in your article
%% use the graphics package for simple commands
%% \usepackage{graphics}
%% or use the graphicx package for more complicated commands
%% \usepackage{graphicx}
%% or use the epsfig package if you prefer to use the old commands
%% \usepackage{epsfig}

%% The amssymb package provides various useful mathematical symbols
\usepackage{amssymb}
%% The amsthm package provides extended theorem environments
%% \usepackage{amsthm}

%% The lineno packages adds line numbers. Start line numbering with
%% \begin{linenumbers}, end it with \end{linenumbers}. Or switch it on
%% for the whole article with \linenumbers after \end{frontmatter}.
\usepackage{lineno}

%% natbib.sty is loaded by default. However, natbib options can be
%% provided with \biboptions{...} command. Following options are
%% valid:

%%   round  -  round parentheses are used (default)
%%   square -  square brackets are used   [option]
%%   curly  -  curly braces are used      {option}
%%   angle  -  angle brackets are used    <option>
%%   semicolon  -  multiple citations separated by semi-colon
%%   colon  - same as semicolon, an earlier confusion
%%   comma  -  separated by comma
%%   numbers-  selects numerical citations
%%   super  -  numerical citations as superscripts
%%   sort   -  sorts multiple citations according to order in ref. list
%%   sort&compress   -  like sort, but also compresses numerical citations
%%   compress - compresses without sorting
%%
%% \biboptions{comma,round}

% \biboptions{}


\journal{my friends}

\begin{document}

\begin{frontmatter}

%% Title, authors and addresses

%% use the tnoteref command within \title for footnotes;
%% use the tnotetext command for the associated footnote;
%% use the fnref command within \author or \address for footnotes;
%% use the fntext command for the associated footnote;
%% use the corref command within \author for corresponding author footnotes;
%% use the cortext command for the associated footnote;
%% use the ead command for the email address,
%% and the form \ead[url] for the home page:
%%
%% \title{Title\tnoteref{label1}}
%% \tnotetext[label1]{}
%% \author{Name\corref{cor1}\fnref{label2}}
%% \ead{email address}
%% \ead[url]{home page}
%% \fntext[label2]{}
%% \cortext[cor1]{}
%% \address{Address\fnref{label3}}
%% \fntext[label3]{}

\title{Quantum Mechanics}

%% use optional labels to link authors explicitly to addresses:
%% \author[label1,label2]{<author name>}
%% \address[label1]{<address>}
%% \address[label2]{<address>}

\author{Joshua Black}

%\address{Address}

%\begin{abstract}
%% Text of abstract
%\end{abstract}

\end{frontmatter}

%%
%% Start line numbering here if you want
%%
%%\linenumbers

%% main text

\section{Introduction}

I'll just say it plain and simple what my plan is for this manuscript. I have a lot of quantum mechanics and qft books. Currently I have 15 QM books and 9 QFT books. These subjects will be very important to my research as I go further into my career, so I decided to condense it all down into one, probably painfully long, paper. The summer going into grad school I read a few thousand pages of qm and the notes for that are all written in paper form. The problem with what I did then was I was taking notes on the same material over and over again. Why not try to condense it down into one set of notes? Also in the future maybe I can use this manuscript as class notes or even have my own set of textbooks? Who knows.

My current plan is to skip the actual introduction stuff that you see at the beginning of ever qm text book. I'm referring to the section on classical mechanics, math methods, and historical context. Its not that those sections aren't important, I just dont know what I need to prep the reader for yet. I'll be making a note of all of that as I go along and then once I am done with the bulk of this paper, I will try to expand on those sections. 

There are certain things that I want to be able to accomplish in this book that sets it apart from all the different textbooks I will be reading. A lot of these textbooks go into topics that are much more useful for chemists. I don't think I will be writing about that here. I want this to be focused on the physics side of things. Even nuclear physics I am not very interested in so very little of that will be explored. Also I find a lot of these books to lack voice. I want my voice to narrate this book instead of just delivering the facts to you. 

Anyways I am going to start write the paper now. At least I have some sort of intro that contains some of the broad ideas of what I am hoping to accomplish.

\section{MM}

Here I am tracking what needs to be done for math methods (mm)

May or may not put at beginning of sections rather than at front. I didnt really enjoy all of that at beginning of books. I want to get to the physics first. 

\section{CM}

Here I am tracking what needs to be done for classical mechanics (cm)

This section may not be in the front of the book. I havent really found that helpful too much. I think putting the relevants parts of cm at the front of each section of the book may be better.


\section{Einstein-Plank Relation and de Broglie Wavelength}

So where I think it is most logical to start would be with the Einstein-Plank relation and the de Broglie wavelength. I like to think of them as partners in crime because of how often they come up together in calculations. I find these relations to be of upmost importance because they lay the building blocks for the rest of quantum mechanics.

To start with, we'll discuss the Einstein-Plank relation. Einstein and Plank are credited with the initial thought of photons, which are particles of light. Plank theorized of them when trying to fix the ultra-violet catastrophe of classical mechanics, and Einstein used the notion of quantized light particles to have a consistent theory for the photoelectric effect. Their relation tell us that the energy of these photons are proportional to their frequency, 

\begin{equation}
    E = \hbar \omega = h \nu.
\end{equation}

The proportionality constant h is called Plank's constant is $6.62607015 \times 10^{-34}$ joule second, while $\hbar$ is referred to as Plank's reduced constant and is equal to $h/2\pi$. That may seem like a very precise measurement, and you also may wonder why I bothered to type it all out. Well like the speed of light we no longer actually measure the value of h but have defined it to be exactly that value in SI units. You should not worry about this value too much because as you progress as a physicist, the more likely you are to run into natural units which just set $\hbar$, the gravitational constant G, Boltzmann's constant $k_B$, and c to 1.

Now onto the de Broglie wavelength. This 



%% The Appendices part is started with the command \appendix;
%% appendix sections are then done as normal sections
%% \appendix

%% \section{}
%% \label{}

%% References
%%
%% Following citation commands can be used in the body text:
%% Usage of \cite is as follows:
%%   \cite{key}          ==>>  [#]
%%   \cite[chap. 2]{key} ==>>  [#, chap. 2]
%%   \citet{key}         ==>>  Author [#]

%% References with bibTeX database:

\bibliographystyle{model1-num-names}
\bibliography{sample.bib}

%% Authors are advised to submit their bibtex database files. They are
%% requested to list a bibtex style file in the manuscript if they do
%% not want to use model1-num-names.bst.

%% References without bibTeX database:

%\begin{thebibliography}{00}

%% \bibitem must have the following form:
%%   \bibitem{key}...
%%

% \bibitem{}

% \end{thebibliography}

\appendix

Appendix

\end{document}

%%
%% End of file `elsarticle-template-1-num.tex'.
